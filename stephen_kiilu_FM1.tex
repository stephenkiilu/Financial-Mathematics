%%%%%%%%%%%%%%%%%%%%%%%%%%%%%%%%%%%%%%%%%%%%%%%%%%%%%%%%%%%%%%%%%%%%%%%%%%%%%%%%
%%%%%%%%%%%%%%%%%%%%%%%%%%%%%%%%%%%%%%%%%%%%%%%%%%%%%%%%%%%%%%%%%%%%%%%%%%%%%%%%
%%% Template for AIMS Rwanda Assignments         %%%              %%%
%%% Author:   AIMS Rwanda tutors                             %%%   ###        %%%
%%% Email: tutors2017-18@aims.ac.rw                               %%%   ###        %%%
%%% Copyright: This template was designed to be used for    %%% #######      %%%
%%% the assignments at AIMS Rwanda during the academic year %%%   ###        %%%
%%% 2017-2018.                                              %%%   #########  %%%
%%% You are free to alter any part of this document for     %%%   ###   ###  %%%
%%% yourself and for distribution.                          %%%   ###   ###  %%%
%%%                                                         %%%              %%%
%%%%%%%%%%%%%%%%%%%%%%%%%%%%%%%%%%%%%%%%%%%%%%%%%%%%%%%%%%%%%%%%%%%%%%%%%%%%%%%%
%%%%%%%%%%%%%%%%%%%%%%%%%%%%%%%%%%%%%%%%%%%%%%%%%%%%%%%%%%%%%%%%%%%%%%%%%%%%%%%%


%%%%%% Ensure that you do not write the questions before each of the solutions because it is not necessary. %%%%%% 

\documentclass[12pt,a4paper]{article}

%%%%%%%%%%%%%%%%%%%%%%%%% packages %%%%%%%%%%%%%%%%%%%%%%%%
\usepackage{amsmath}
\usepackage{amssymb}
\usepackage{amsthm}
\usepackage{amsfonts}
\usepackage{graphicx}
\usepackage[all]{xy}
\usepackage{tikz}
\usepackage{placeins}
\usepackage{actuarialsymbol}
\usepackage{actuarialangle}
\usepackage{float}
\usepackage{verbatim}
\usepackage[left=2cm,right=2cm,top=3cm,bottom=2.5cm]{geometry}
\usepackage{hyperref}
\usepackage{caption}
\usepackage{subcaption}
\usepackage{psfrag}

%%%%%%%%%%%%%%%%%%%%% students data %%%%%%%%%%%%%%%%%%%%%%%%
\newcommand{\student}{Stephen Kiilu}
\newcommand{\course}{Financial Mathematics}
\newcommand{\assignment}{FM1}

%%%%%%%%%%%%%%%%%%% using theorem style %%%%%%%%%%%%%%%%%%%%
\newtheorem{thm}{Theorem}
\newtheorem{lem}[thm]{Lemma}
\newtheorem{defn}[thm]{Definition}
\newtheorem{exa}[thm]{Example}
\newtheorem{rem}[thm]{Remark}
\newtheorem{coro}[thm]{Corollary}
\newtheorem{quest}{Question}[section]

%%%%%%%%%%%%%%  Shortcut for usual set of numbers  %%%%%%%%%%%

\newcommand{\N}{\mathbb{N}}
\newcommand{\Z}{\mathbb{Z}}
\newcommand{\Q}{\mathbb{Q}}
\newcommand{\R}{\mathbb{R}}
\newcommand{\C}{\mathbb{C}}

%%%%%%%%%%%%%%%%%%%%%%%%%%%%%%%%%%%%%%%%%%%%%%%%%%%%%%%555
\begin{document}

%%%%%%%%%%%%%%%%%%%%%%% title page %%%%%%%%%%%%%%%%%%%%%%%%%%
\thispagestyle{empty}
\begin{center}
\textbf{AFRICAN INSTITUTE FOR MATHEMATICAL SCIENCES \\[0.5cm]
(AIMS RWANDA, KIGALI)}
\vspace{1.0cm}
\end{center}

%%%%%%%%%%%%%%%%%%%%% assignment information %%%%%%%%%%%%%%%%
\noindent
\rule{17cm}{0.2cm}\\[0.3cm]
Name: \student \hfill Assignment Number: \assignment\\[0.1cm]
Course: \course \hfill Date: \today\\
\rule{17cm}{0.05cm}
\vspace{1.0cm}


\section*{Question 1}

In this exercise we are given force of interest per annum earned on bank deposits in a certain year. The woman deposits $\pounds$ 10,000 and makes withdrawals of $\pounds$ 1,000 at end of each quarter.
\begin{enumerate}
\item[(a)]
Calculate the closing balance of the woman's account.\\
We use accumulation function and force of interest at different proportions of the year to determine the woman's closing account. The figure \ref{fig 1} below shows the deposit and withdrawals made at each quarter.
\begin{figure}[H]
\includegraphics[width=12cm]{2}
\centering
\caption{accumulation of bank deposits}
\label{fig 1}
\end{figure}
\begin{eqnarray*}
u\left(0,\frac{1}{4} \right)&=& e^{\int_0 ^{\frac{1}{4}}(0.05+0.04t)dt}=1.0138\\
u\left(\frac{1}{4},\frac{1}{2} \right)&=& e^{\int_{\frac{1}{4}} ^{\frac{1}{2}}(0.05+0.04t)dt}=1.01638\\
u\left(\frac{1}{2},\frac{3}{4} \right)&=& e^{\int_{\frac{1}{2}} ^{\frac{3}{4}}(0.07-0.04(t-0.5)^2 )dt}=1.0174\\
u\left(\frac{3}{4},1 \right)&=& e^{\int_{\frac{3}{4}} ^1 (0.07-0.04(t-0.5)^2 )dt}=1.01617\\
\end{eqnarray*}
The amount accumulated after the first quarter is given by;
\begin{eqnarray*}
&=&10,000 \,u\left(0,\frac{1}{4} \right)-1000\\
&=&10,000 \times 1.0138-1000\\
&=& \pounds 9,138.
\end{eqnarray*}
The amount accumulated after the second quarter;
\begin{eqnarray*}
&=& 9138 \,u\left(\frac{1}{4},\frac{1}{2} \right)-1000\\
&=&9,138 \times 1.01638-1000\\
&=& \pounds 8,287.6804.
\end{eqnarray*}

The amount accumulated after the third quarter;
\begin{eqnarray*}
&=&8287.6804\, u\left(\frac{1}{2},\frac{3}{4} \right)-1000\\
&=& 8287.6804\times 1.0174-1000\\
&=& \pounds 7,431.88608.
\end{eqnarray*}
The closing balance at the end of the year;
\begin{eqnarray*}
&=& 7431.88608\, u\left(\frac{3}{4},1 \right)-1000\\
&=& 7431.88608\times 1.01617-1000\\
&=& \pounds 7,552.06
\end{eqnarray*}
\item[(b)]
Find the nominal rate of interest per annum
\begin{eqnarray*}
i_h(t)&=&\frac{u(t,t+h)-1}{h}\\
h&=&\frac{1}{2}, \,
t=\frac{1}{2}\\
i_h(t)&=& \frac{u(1/2,1)-1}{1/2}\\
&=& \frac{e^{\int_{\frac{1}{2}} ^{1}(0.07-0.04(t-0.5)^2 )dt}-1}{1/2}\\
&=& 0.06778\\
&=& 6.78 \%
\end{eqnarray*}
\end{enumerate}

\section*{Question 2}
\begin{enumerate}
\item[(a)] We are required to show that;
\begin{eqnarray*}
a_{\angl{n}}& =&\frac{1-v^n}{i}\\
\text{and we know that;}\\
u& =&i+1\\
v&=&i-d\\
uv&=&1\\
a_{\angl{n}}& =&v\left(\frac{1-v^n}{1-v}\right)\\
a_{\angl{n}}& =&\frac{1}{1+i}\left(  \frac{1-v^n}{1-\frac{1}{1+i}}\right)\\
&=&\frac{1}{1+i} \left(   \frac{1-v^n}{1} 1+i\right)\\
\text{ and therefore we have}\\
a_{\angl{n}}& =&\frac{1-v^n}{i}\\
\text{As required.}
\end{eqnarray*}

We need to show that;
\begin{eqnarray*}
s_{\angl{n}}& =&(1+i)^n a_{\angl{n}}\\
s_{\angl{n}}&=& \left( \frac{\frac{1}{v^n}-1}{u-1}\right)\\
&=&\left( \frac{\frac{1-v^n}{v^n}}{u-1}\right)\\
&=&\left( \frac{1-v^n}{v^n}.\frac{1}{u-1}\right)\\
&=& \frac{1-v^n}{v^n}.\frac{1}{i}\\
&=& \frac{1-v^n}{i}.\frac{1}{v^n}\\
&=& \frac{1-v^n}{i}. u^n\\
\text{ but } \, u=1+i\\
&=& (1+i)^n a_{\angl{n}}\\
\text{and therefore}\\
s_{\angl{n}}& =&(1+i)^n a_{\angl{n}}\\
\text{as required.}
\end{eqnarray*}
We want show that;
\begin{eqnarray*}
\frac{1}{a_{\angl{n}}}-\frac{1}{s_{\angl{n}}} &=& i\\
a_{\angl{n}}-s_{\angl{n}} &=& \frac{i}{v^n}-\frac{1}{(1+i)^n}.\frac{i}{1-v^n}\\
 &=&\frac{i}{1-v^n}- \frac{i}{u^n(1-v^n)}\\
 &=& \frac{i(u^n-1}{u^n-(uv)^n}\\
  &=&\frac{i(u^n-1)}{u^n-1}\\
   &=& i\\
   \text{and therefore}\\
\frac{1}{a_{\angl{n}}}-\frac{1}{s_{\angl{n}}} &=& i\\
   \text{as required.}
\end{eqnarray*}
\item[(b)]
We required to find i and n given that,
\begin{eqnarray*}
s_{\angl{n}}=14.2068 \,\, ,a_{\angl{n}}=8.3064\\
\frac{1}{a_{\angl{n}}}-\frac{1}{s_{\angl{n}}} &=& i\\
&=&\frac{1}{8.3064}-\frac{1}{14.2068}\\
&=& 0.05\\
&=& 5\%
\end{eqnarray*}
We now calculate the value of n;
\begin{eqnarray*}
\frac{s_{\angl{n}}}{a_{\angl{n}}}&=& (i+1)^n\\
n&=&\frac{\ln \left( \frac{s_{\angl{n}}}{a_{\angl{n}}} \right)}{\ln (i+1)}\\
n&=&\frac{\ln \left( \frac{14.2068}{8.3064}\right)}{ \ln (1.05)}\\
n&=& 11.
\end{eqnarray*}
\end{enumerate}
\section*{Question 3}
We need to get an expression for present value of loan.\\
Loan is $ \pounds 5000$ , let the initial amount paid be x, and the loan is payable in 15 years.
The figure \ref{fig 3} below shows the annuity paid each year for period of 15 years.
\begin{figure}[H]
\includegraphics[width=14cm]{3}
\centering
\caption{annuties for 15 years}
\label{fig 3}
\end{figure}
\begin{enumerate}
\item[(a)]
Payment of x amount for the 15 years.
\begin{eqnarray*}
xv^1+xv^2+\cdots +xv^{15}&=&x a_{\angl{15}}
\end{eqnarray*}
Payment of \pounds 100 for the 10 years.
\begin{eqnarray*}
100v^6+100v^7+\cdots +100 v^{15}&=&100(a_{\angl{15}}-a_{\angl{5}})
\end{eqnarray*}
Payment of \pounds 200, after 10 years.
\begin{eqnarray*}
200v^{11}+200v^{12}+\cdots + 200v^{15}&=&200(a_{\angl{15}}-a_{\angl{10}})
\end{eqnarray*}
Our expression becomes;
\begin{eqnarray*}
5000- x a_{\angl{15}}-100(a_{\angl{15}}-a_{\angl{5}})-200(a_{\angl{15}}-a_{\angl{10}})
\end{eqnarray*}
For us to calculate the initial amount we equate the expression above to zero and solve for x.
\begin{eqnarray*}
0=5000- x a_{\angl{15}}-100(a_{\angl{15}}-a_{\angl{5}})-200(a_{\angl{15}}-a_{\angl{10}})
\end{eqnarray*}
\begin{eqnarray*}
a_{\angl{n}}&=&\frac{1-(1+i)^{-n}}{i}\\
a_{\angl{15}}&=& \frac{1-(1.04)^{-15}}{0.04} = 11.1184\\
a_{\angl{10}}&=& \frac{1-(1.04)^{-10}}{0.04} = 8.1109\\
a_{\angl{5}}&=& \frac{1-(1.04)^{-5}}{0.04} = 4.4518\\
0&=& 5000-11.1184x-601.5-666.66\\
x&=&\pounds 335.65\\
\text{which is the initial amount of the annual payment.}
\end{eqnarray*}
We now calculate the loan after third year
\begin{eqnarray*}
u&=&1+i\\
u&=&1.04\\
&&5000 u^3-(xu^2+xu+x)\\
 \text{and}\, \,x&=& 335.65\\ \text{which gives loan after $3^{rd}$ year as;}\\
&&\pounds 4576.55\\
\text{interest paid} &=& 4/100 \times 4576.55\\
&&\pounds 183.06\\
\end{eqnarray*}
\item[(c)]
We need to calculate level payments if the loan was to be recalculated after 7 years. We first calculate the outstanding balance and the calculate the uniform payments. The figure \ref{fig 5} shows level payment of the loan after 7th year.
\begin{figure}[H]
\includegraphics[width=10cm]{4}
\centering
\caption{uniform payments}
\label{fig 5}
\end{figure}
We first calculate the outstanding loan, z
\begin{eqnarray*}
z&=& 5000u^7-(xu^6+xu^5+xu^4+xu^3+xu^2+(x+100)u+(x+100))\\
\end{eqnarray*}
Replacing for the values of z and x in the above equation, we get;
\begin{eqnarray*}
z=6579.6589-(424.54+408.21+392.51+377.41+362.90+452.94+435.65)\\
\end{eqnarray*}
The outstanding balance is;
\begin{eqnarray*}
=\pounds 3725.49\\
\end{eqnarray*}
We know calculate level payments after 7th year, y
\begin{eqnarray*}
z&=& y.\frac{1-v^n}{i}\\
z&=&y.\frac{1-(1+i)^n}{i}\\
y&=&\frac{zi}{1-(1+i)^{-n}}\\
y&=& \frac{3725.49 \times 0.04}{1-(1.04)^{-8}}\\
&=& \pounds 553.34.
\end{eqnarray*}
\end{enumerate}
\section*{Question 4}
The figure \ref{fig 6} shows discounted payback period of a project at an interest of $7.5 \%$.
\begin{figure}[H]
\includegraphics[width=10cm]{5}
\centering
\caption{discounted payback}
\label{fig 6}
\end{figure}
\begin{enumerate}
\item[(a)]
We need to show that the discounted payback period, t,
of the project is given by;
\begin{eqnarray*}
t&=& 2-\frac{\ln \left[1-2\delta(1+i)(4+3i) \right]}{\delta}\\
\end{eqnarray*}
Here we use accumulation function;
\begin{eqnarray*}
150,000 u^2+50,000 u&=&25,000 \int _0 ^{t-2} e^{-\delta t}dt\\
6 u^2+2 u&= &\int _0 ^{t-2} e^{-\delta t }dt\\
\text{but} \, u=1+i\\
6u^2+2u&=& 6(1+i)^2+2(i+1)= 2(i+1)(3(1+i)+1)\\
&=& 2(i+1)(4+3i)\\
\int _0 ^{t-2} e^{-\delta t}dt&=&\frac{1-e^{-\delta(t-2)}}{\delta}\\
 2(i+1)(4+3i)&=& \frac{1-e^{-\delta(T-2)}}{\delta}\\
 e^{-\delta(t-2)}&=&1-2 \delta  2(i+1)(4+3i)\\
 \text{we introduce natural log on both sides to obtain; }\\
 -\delta(t-2)&=& \ln(1-2 \delta  2(i+1)(4+3i))\\
 t-2 &=& -\frac{\ln \left[1-2\sigma(1+i)(4+3i) \right]}{\delta}\\
 t&=& 2-\frac{\ln \left[1-2\sigma(1+i)(4+3i) \right]}{\delta}\\
\end{eqnarray*}
\newpage
Evaluate this expression at the specified interest;
\begin{eqnarray*}
\text{We are given}\\
i&=& 0.075\\
\delta &=& \ln (1+i)= \ln (1.075)\\
\text{We substitute this figures in ;}\\
 t&=& 2-\frac{\ln \left[1-2\sigma(1+i)(4+3i) \right]}{\delta}\\
 \text{to obtain t as}\\
  t&=& 2-\frac{\ln \left[1-2\ln(1.075)(1.075)(4+3(0.075)) \right]}{\ln(1.075)}\\
 \text{to obtain t as}\\
 t&=& 16.79.
\end{eqnarray*}
 \item[(b)]
 We want to find the accumulated amount of this
account 25 years after purchase of the property.
\begin{eqnarray*}
\bar{s}_{\angl{n}}&=&\frac{u^n-1}{\delta}\\
\text{we are given;}\\
u= 1+i \,\, ,\,
i&=& 0.05\,\, , \,
n= 8\\
\delta = \ln (u)\\
25,000 \bar{s}_{\angl{n}}&=&25,000\left(\frac{u^n-1}{\delta}\right)\\
25,000\left(\frac{u^8-1}{\delta}\right)&=&25,000\left(\frac{1.05^8-1}{\ln(1.05)}\right)\\
&=& \pounds 244,647.39.
\end{eqnarray*}
\end{enumerate}
\newpage
\begin{thebibliography}{99}

  \bibitem{notes} Prof. James Vickers  ,{\em Lecture notes,}  2021.\\
%
%  \bibitem{notes} Babatunde Abiodun {\em Lecture notes,}  2021.
%  
%  \bibitem{norman} Ronald Stull, { 
%Meteorology for Scientists and Engineers, 3rd Ed.} Chapter 1, THE ATMOSPHERE: 2011, 2015 Copyright@2011
%
\end{thebibliography}

\end{document}